\documentclass{article}
\usepackage{graphicx} % Required for inserting images
\usepackage{ctex}
\usepackage{enumitem}
\usepackage{amsmath} 
\usepackage[a4paper, margin=1in]{geometry}
\title{考博}
\author{MK MK}
\date{\today}


\begin{document}

\maketitle

\section{原子核物理}


\begin{enumerate}
    \item 题目:\textbf{给出三种$\beta$衰变的反应式及能量条件.}\\答案:《原子核物理》p135
\item 题目:\textbf{$p + ^{7}Li \rightarrow 2^{4}He + 17.6 MeV$。入射P能量1MeV,生成两个$\alpha$粒子出射方向沿p入射方向对称分布,求$\alpha$粒子能量及出射角$\theta$}\\
 答案:入射质子与静止的锂-7核发生反应生成两个$\alpha$粒子,释放17.6 MeV能量。根据能量守恒,总动能为质子动能(1 MeV)与释放能量之和:

\[
K_{\text{总}} = 1\ \text{MeV} + 17.6\ \text{MeV} = 18.6\ \text{MeV}
\]

两个$\alpha$粒子动能均分,故每个$\alpha$粒子的能量为:

\[
K_{\alpha} = \frac{18.6}{2} = 9.3\ \text{MeV}
\]

动量守恒方面,质子动量 \(p_p = \sqrt{2m_p K_p}\),$\alpha$粒子动量 \(p_{\alpha} = \sqrt{2m_{\alpha} K_{\alpha}}\)。由于对称性,两$\alpha$粒子横向动量分量抵消,纵向分量满足:

\[
2p_{\alpha} \cos\theta = p_p
\]

代入质量关系 \(m_{\alpha} \approx 4m_p\),解得:

\[
\cos\theta = \frac{\sqrt{m_p K_p}}{2\sqrt{4m_p \cdot 9.3}} = \frac{\sqrt{1}}{2\sqrt{4 \cdot 9.3}} = \frac{1}{2\sqrt{37.2}} \approx 0.082
\]

因此,出射角为:

\[
\theta = \arccos(0.082) \approx 85.3^\circ
\]

**答案:**

每个$\alpha$粒子的能量为 \(\boxed{9.3\ \text{MeV}}\),出射角为 \(\boxed{85.3^\circ}\)。
\item 

\end{enumerate}





\end{document}
